%\documentclass[12pt]{article}
\documentclass[espaco=simples,appendix=Name]{abnt}
\usepackage{abntex}
\usepackage{multirow} 
\usepackage[brazil]{babel}
\usepackage[T1]{fontenc}
\usepackage[utf-8]{inputenc}
\usepackage{hyperref}
\usepackage{times}
\usepackage{listings}
\usepackage[dvips]{graphicx}
\usepackage[num]{abntcite}      % citacoes do abntex
\usepackage{tabela-simbolos}    % tabelas de simbolos do abntex
\usepackage{dsfont}             % fonte
\usepackage{fancyvrb}

\citeoption{abnt-full-initials=yes}
\lstset{language=Ruby,caption=Exemplo,label=Ruby, numbers=left, frame=single} 

\title{Estatística aplicada à linguagem de programação Ruby}

\author{Jônatas Davi Paganini}

\date{fevereiro de 2010}

\begin{document}

\maketitle

\chapter{Introdução}

A eminência da estatística, realmente comprova que ela está na moda. O profissional desta área têm conquistado um espaço cada vez maior. Através das fórmulas, é possível visualizar uma série de informações. Na prática é possível distribuir, classificar e sintetizar dados de diferentes formas e ângulos. Cada disposição dos dados pode revelar diversos tipos de informação, serve ao estatístico, analisar e encontrar os significados e respostas esperados.

Um resultado estatístico é composto por uma série de cálculos e disposições que tornam possível chegar aquele resultado. Quando existe uma quantidade relevante de dados e passos, o trabalho do estatístico se torna mais árduo, se não tornar-se automatizado.

Este trabalho é composto por duas raízes básicas de estudo: A estatística aplicada e programação básica na linguagem Ruby. O primeiro busca explicar as fórmulas e exibir os passos para chegar a um determinado resultado. A segunda tem como objetivo automatizar o processamento das fórmulas e passos realizados anteriormente.

O estudo da estatística é simples e objetivo, cada dado é captado por uma pesquisa com algum objetivo, é executado em um determinado tempo e local. Em outras palavras, uma pesquisa deve ser composta pelos elementos básicos: O quê, quando e quem.

Através do estudo da estatística, exemplo de cálculos práticos, este artigo procura abordar o uso da linguagem de programação Ruby como ferramenta para automatizar os cálculos estatísticos e codificação das fórmulas e decomposição do raciocíonio lógico. Através dos examplos práticos de codificação, cada assunto será codificado, e os novos elementos da linguagem de programação Ruby, serão explicados, após o seu uso, tornando o ensino da programação, uma formalidade do assunto precedido.

Este estudo pretende abordar o conteúdo de forma suscinta, tornado a linguagem de programação apenas uma ferramenta simples para ajudar a automatizar o processo. Desta maneira, é possível experimentar diversas abordagens da linguagem de programação para resolução dos problemas.  


\chapter{Estatística descritiva}

A estatística descritiva, é a técnica usada para resumir, comparar, observar e descrever informações relevantes de um ou mais conjuntos de dados. Um conjunto de dados pode ser análisado básicamente em dois passos: organização e método.

Segundo Jairo e Gilberto \cite{cursoEstatistica}, a estatística descritiva, como o próprio nome sugere, se constitui num conjunto de técnicas que objetivam descrever, analisar e interpretar os dados numéricos de uma população ou amostra.


\section{ Distribuição de frequência } 

\subsection { Na estatística }
Dado uma pesquisa realizada na turma do último ano de sistemas de informação, realizada no ano de 2010, buscando saber mais informações a respeito da faixa etária da turma, foram obtidas as seguintes idades:

23, 31, 31, 21, 31, 26, 31, 22, 22, 24, 31, 21, 22, 20, 22, 31, 21, 20

A partir das idades (dados) coletados acima, serão analisados e classificados os dados. Os dados acima exibidos, estão na forma bruta, ou seja, aleatória e desorganizada. 



Em Ruby, podemos representar os mesmos dados brutos através de um vetor de números, e podem ser atribuídos a uma variável chamada \textbf{dados\_brutos}.


\lstinputlisting[firstline=1,
                  lastline=2,
                   caption=Atribuindo dados brutos a uma variável]{estatistica.rb} 


No exemplo acima, dados\_brutos é o nome da variável que recebe o vetor de números. O operador \textbf{=}(igual) identifica que a variável dados\_brutos é igual ao vetor declarado posteriormente.

Um vetor ou \textit{Array} é delimitado pelo compilador Ruby pelos caractéres de abertura de colchetes para iniciar os elementos, e fechamento de colchetes para declarar o fim dos elementos.

A quebra de linhas e espaços em branco não fazem diferença alguma para o compilador, os dados acima estão dispostos em duas linhas para ficar mais legível para estudo.


\section { Análise e classificação dos dados }

Para entender e análisar os dados com mais facilidade, o primeiro passo é ordenar os elementos por ordem crescente ou decrescente. Neste caso, usaremos ordem crescente. 


\subsection { Rol }

Este método de organização está dos dados em ordem crescente é também conhecido como Rol. Organizando então o Rol dos dados obtidos na pesquisa acima tem-se:

20, 20, 21, 21, 21, 22, 22, 22, 22, 23, 24, 26, 31, 31, 31, 31, 31, 31

\subsection { Implementando Rol em Ruby }

Seguindo a linha do código obtido anteriormente, em Ruby, é possível obter o mesmo resultado fácilmente ordenando os valores do vetor. Nativamente, Ruby já tráz este método de ordenação e o nome dele é sort.

\lstinputlisting[firstline=3, lastline=3,
                 caption=Atribuindo dados brutos a uma variável]{estatistica.rb} 

Neste caso, analisando o código acima descrito, têm-se uma váriavel \textbf{rol} recebendo o valor do método sort, que ordena os \textbf{dados\_brutos}.

Inspecionando o resultado da variável \textbf{rol}, temos o seguinte vetor:


\lstinputlisting[firstline=4, lastline=4,
                 caption=Inspecionando a variável rol]{estatistica.rb} 

Através do método \textbf{p} é possível imprimir em tela os valores de um objeto em Ruby. Este método tem o objetivo de ajudar o programador a entender como funciona o código e visualizar as propriedades de uma variável. No exemplo anterior, foi usado para visualizar se a ordem das idades que estão na variável \textbf{rol} está correta.


\subsection { Frequência absoluta (fi) }




Sabendo que o rol já está organizado, é possível inserir  as frequências absolutas dos dados. A primeira informação a se extrair do \textbf{rol} é a \textbf{frequência absoluta}, também caracterizado pela sigla \textbf(fi). O \textbf{fi} é determinada pelo número de repetições de cada dado da consulta. Neste exemplo, é dada pela quantidade de alunos com a mesma idade.

Desta forma teríamos a seguinte classificação:

20, 20

21, 21, 21

22, 22, 22, 22

23

24

26

31, 31, 31, 31, 31, 31


Conforme os números acima, é possível imaginar que a frequência absoluta dos alunos com 20 anos de idade é 2, assim como, 21 é 3, 26 é 1 e existem 6 alunos com 31 anos.

Entendendo que o objetivo deste cálculo é contar os dados com mesmo valor, já é possível avançar este passo na programação.

\subsection { Programando a frequência absoluta (fi) }

Sabendo que a variável \textbf{rol} possuí as idades ordenadas, existem várias formas de implementar a solução. A primeira forma, aborda a contagem de elementos de cada tipo, e o amarzenamento dele é feito através de um \textbf{Hash}. Este tipo de elemento, funciona como uma estrutura, que permite adicionar chaves com um determinado valor. Esta chave, pode ser um objeto de qualquer tipo, e também é possível recuperar ou estabelecer um valor para uma chave.

O algorítmo que será criado, consiste em criar um \textbf{Hash} ou seja, uma variável que possua várias chaves e valores. E cada chave, será representada pela própria idade, e o valor, será a quantidade de vezes que aquela idade passou está presente no \textbf{rol}. 

Se não existir um \textbf(fi) para aquela idade, então será adicionado o valor 1 para aquela idade.
Caso contrário, se encontrar a idade, será adicionado mais 1 na frequência absoluta desta idade.

\lstinputlisting[firstline=5, lastline=12, label=fi,
                 caption=criando o fi]{estatistica.rb} 

O código acima é iniciado pela declaração de uma nova variável, que se chama \textbf{fi} e sabe-se que é do tipo \textbf{Hash}, pois é inicializada com chaves(\textbf{\{\}}). 

Logo na linha 2, é iniciado uma iteração(método \textbf{each}) com cada \textbf{idade} do \textbf{rol}. Esta variável (\textbf{idade}) é declarada após o método \textbf{each} e reconhecida pelo \textbf{bloco} delimitado por \textbf{do} e \textbf{end}(linhas 6 e 12). As variáveis de bloco, declaradas após \textbf{do}(linha 6) só funcionam dentro do bloco(\textbf{do..end}) e são interpoladas pelo caracter \textbf{|}.

Para simplificar a explicação, é possível acompanhar na prática como este código funciona. A variável em foco neste momento, é \textbf{fi}, ou seja, o \textbf{Hash} de \textbf{idades} com suas respectivas \textbf{quantidades}.
 Usando o método \textbf{p} na variável \textbf{fi}, é possível acompanhar passo a passo, as frequências absolutas sendo somadas.

Uma simples inspeção pode ser adicionada, como neste fragmento de código:

\begin{lstlisting}[caption=Inspecionando a soma dos elementos, emph=2]
  p fi
\end{lstlisting}

Se o código acima for adicionado após a linha 6 da listagem do fi \ref{fi}, então irá imprimir as seguintes linhas:

\begin{lstlisting}[caption=inspeção da variável \textbf{fi} ]
{}
{20=>1}
{20=>2}
{20=>2, 21=>1}
{20=>2, 21=>2}
{20=>2, 21=>3}
{22=>1, 20=>2, 21=>3}
{22=>2, 20=>2, 21=>3}
{22=>3, 20=>2, 21=>3}
{22=>4, 20=>2, 21=>3}
{22=>4, 23=>1, 20=>2, 21=>3}
{22=>4, 23=>1, 24=>1, 20=>2, 21=>3}
{22=>4, 23=>1, 24=>1, 20=>2, 26=>1, 21=>3}
{22=>4, 23=>1, 24=>1, 31=>1, 20=>2, 26=>1, 21=>3}
{22=>4, 23=>1, 24=>1, 31=>2, 20=>2, 26=>1, 21=>3}
{22=>4, 23=>1, 24=>1, 31=>3, 20=>2, 26=>1, 21=>3}
{22=>4, 23=>1, 24=>1, 31=>4, 20=>2, 26=>1, 21=>3}
{22=>4, 23=>1, 24=>1, 31=>5, 20=>2, 26=>1, 21=>3}
{22=>4, 23=>1, 24=>1, 31=>6, 20=>2, 26=>1, 21=>3}
\end{lstlisting}

Observe como na primeira iteração o \textbf{fi} está vazio, e logo a idade \textbf{20} (linha 2) ganha a primeira quantidade \textbf{1}. Na segunda vez, o \textbf{fi} para idade 20 já existe, então acrescenta-se mais \textbf{1} ao \textbf{fi} desta \textbf{idade}(executa o código do \textbf{else}). Na próxima linha o ciclo se inicia novamente com a \textbf{idade 21}, fazendo com que a variável \textbf{fi} ganhe uma quantidade 1 para a idade 21.


\subsection { Entendendo o funcionamento de um Hash }

A classe \textbf{Hash} não possúi ordenação, por este motivo, a impressão dos elementos acima, não manteve a mesma sequência entre as idades e as quantidades respectivas, mas manteve consistência nos valores.

Este tipo de objeto também pode ser declarado da seguinte forma:

\begin{lstlisting}[caption=Sintaxe de declaração de um \textbf{Hash}]
aluno = { 
  :nome => "jonatas",
  :idade => 23
}
\end{lstlisting}

No código acima, uma variável chamada aluno foi declarada, esta variável é um tipo de \textbf{Hash}, e é delimitada por abertura de chaves \textbf{\{} e fechamento de chaves \textbf{\}}. Os elementos internos como \textbf{:nome} e \textbf{:idade} são conhecidos como chaves, e \textbf{"jonatas"} e 23 são seus respectivos valores. Desta forma é possível acessar os valores através das chaves, e as chaves devem ficar entre colchetes, conforme o exemplo abaixo. 


\begin{lstlisting}[caption=Usufruindo dos métodos do \textbf{Hash}]
puts aluno[:nome] 
aluno[:idade] += 10
\end{lstlisting}

A primeira linha do código anterior imprime o nome do aluno(\textbf{jonatas}), enquanto a segunda, adiciona mais 10 anos a chave \textbf{:idade}. O mesmo exemplo da segunda linha poderia ser escrito da seguinte forma:

\begin{lstlisting}[caption=Somando 10 anos a chave :idade]
aluno[:idade] = aluno[:idade] + 10
\end{lstlisting}

Desta forma é possível deduzir que uma atribuição de uma nova chave em um \textbf{Hash}, é da mesma forma como a concatenação de um valor.

\begin{lstlisting}[caption=Atribuindo o valor 23 para a chave :idade da variável aluno]
aluno[:idade] = 23 
\end{lstlisting}

\section { Tabelas Estatísticas }

Existem várias exigências para a criação de uma tabela estatística, e estas, se encaixam em uma estrutura de:



\begin{itemize} 
  \item{cabeçalho}
  \item{corpo}
  \item{rodapé}
\end{itemize} 

A tabela estatística, Com estes três elementos descritos de forma coerente, é possível ler os dados de uma tabela sem esforço. 

O primeiro passo, é declarar o cabeçalho, que deve conter informações suficientes para responder \textbf{o quê}, \textbf{quando} e \textbf{onde}. Deve primeiramente descrever \textbf{o quê} está demonstrando, deve apresentar uma data ou ano explicando \textbf{quando} ocorreu a pesquisa ou fato. E por último, deve apresentar a fonte dos dados, de \textbf{onde} eles vieram.

\begin{table}
\caption{Pesquisa da idade dos alunos do curso do quarto ano de Sistemas de Informação da Unipar de Francisco Beltrão, no ano de 2010.}
\begin{tabular}{|l|c|} 
\hline 
Idade & Frequência Absoluta (\textbf{fi}) \\ \hline 
20 & 2 \\
21 & 3 \\
22 & 4 \\
23 & 1 \\
24 & 1 \\
26 & 1 \\
36 & 6 \\ 
\hline 
\end{tabular}
\small{Fonte: Unipar}
\end{table}

\bibliographystyle{plainat}
\begin{thebibliography}{9}

\bibitem{cursoEstatistica}
Simon da Fonseca, Jairo; Martins, Gilberto de Andrade.
Curso de Estatística.
Sexta edição, São Paulo, SP. Editora ATLAS S.A., 1996.
\end{thebibliography} 


\end{document}


