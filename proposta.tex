%\documentclass[12pt]{article}
\documentclass[espaco=simples,appendix=Name]{abnt}
\usepackage{abntex}
\usepackage[brazil]{babel}
\usepackage[T1]{fontenc}
\usepackage[utf-8]{inputenc}
\usepackage{hyperref}
\usepackage{times}
\usepackage{listings}
\usepackage[dvips]{graphicx}
\usepackage[num]{abntcite}      % citacoes do abntex
\usepackage{tabela-simbolos}    % tabelas de simbolos do abntex
\usepackage{dsfont}             % fonte
\usepackage{fancyvrb}

\citeoption{abnt-full-initials=yes}
\lstset{language=Ruby,caption=Exemplo,label=Ruby, numbers=left, frame=single} 

\title{Proposta para artigo científico sobre a estatística aplicada a linguagem de programação Ruby}

\author{Jônatas Davi Paganini}

\date{fevereiro de 2010}

\begin{document}

\maketitle

\chapter{Proposta}

A proposta é descrever a linguagem de programação Ruby aplicada ao estudo da estatística e o estudo da estatística aplicado a linguagem de programação Ruby. 

Através do estudo da estatística, exemplo de cálculos práticos, este artigo procura abordar o uso da linguagem de programação Ruby como ferramenta para automatizar os cálculos estatísticos e codificação das fórmulas e decomposição do raciocíonio lógico. Através dos examplos práticos de codificação, cada assunto será codificado, e os novos elementos da linguagem de programação Ruby, serão explicados, após o seu uso, tornando o ensino da programação, uma formalidade do assunto precedido.


\chapter{Motivação}

A maioria dos estudos da linguagem de programação são genéricos e não conseguem ter uma aplicação, atingir um objetivo por completo no ensino de programação. Ao aprender a programar,  o aluno encara muitas explicações abstratas, sobre algo que nunca viu e nem é aplicável ao seu aprendizado.

Este estudo pretende abordar o conteúdo de forma suscinta, tornado a linguagem de programação apenas uma ferramenta simples para ajudar a automatizar o processo. Desta maneira, é possível experimentar diversas abordagens da linguagem de programação para resolução dos problemas.  


\chapter{Métodologia}

Ao desenvolver ou após cada aula, a apresentação do conteúdo será sintetizada e revista, conforme a aula dada, e será desenvolvido um programa que executa o mesmo cálculo e satisfaz os cálculos apresentados em aula.

Ao iniciar a próxima aula, será apresentado o conteúdo produzido na aula anterior e também a solução codificada do exemplo. A professora, irá conferir o conteúdo e está aberta a fazer modificações ou sugestões. 

Na conferência do conteúdo produzido, a professora deve estar ciente de que entendeu como o raciocínio lógico foi codificado e através das explicações sobre cada linha de código, terá o acompanhamento da codificação e evolução dos cálculos e do programa. 

Através das explicações, também deve ser possível deduzir outros cálculos de mesmo natureza e evoluir as fórmulas já existentes.

A cada bimestre ou conclusão de módulos de estudo, será compilado um material para publicação.

\end{document}
